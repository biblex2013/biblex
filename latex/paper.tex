\documentclass[11pt]{article}

\usepackage{amsfonts}
\usepackage{amsthm}
\usepackage{amsmath}
\usepackage{amssymb}
\usepackage[utf8x]{inputenc}
\usepackage[english]{babel}
\usepackage{cite}

\title{Three easy steps to manage your references}
\date{April 8, 2013}
\author{The \textit{biblex} team}

\begin{document}
\maketitle

\begin{abstract}
This abstract is kinda abstract.
\end{abstract}

\section{Introduction} Before managing your references, one must realize that doing that with bib\TeX is much easier than deducing the closed form of
\[\sum_{i = 1}^n i^2,\]
as outlined by \cite{biblex13}. Also \cite{millennium01} concluded the same.

\bibliographystyle{plain}
\bibliography{bibliography}{}
\end{document}